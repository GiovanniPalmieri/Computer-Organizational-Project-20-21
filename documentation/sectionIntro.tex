\section{Introduzione}
Il progetto prevede l'implementazione di una lista concatenata doppia. Ogni nodo 
della lista è strutturato secondo la Tabella~\ref{tab:nodeStructure}. Nel caso in cui un nodo non abbia elemento successivo o precedente, PBACK o PAHEAD 
vengono impostati al valore -1 (0xFFFFFFFF in complemento a due).
\begin{table}[H]
    \begin{center}
    \begin{tabular}{|c|c|c|}
        \hline
        Nome & Dimensione & Descrizione \\
        \hline
        PBACK & 4 byte & Puntatore all'elemento precedente \\
        \hline
        DATA & 1 byte & Informazione del nodo \\
        \hline
        PAHEAD & 4 byte & Puntatore all'elemento successivo \\
        \hline
    \end{tabular}
    \caption{Struttura nodi}
    \label{tab:nodeStructure}
    \end{center}
\end{table}  

La lista prevede 5 operazioni:
\begin{itemize}
    \item ADD (char)\@: Aggiunta di un nuovo elemento in coda alla lista con DATA=char.
    \item DEL (char)\@: Rimuove (se presente) il primo elemento della lista con DATA=char.
    \item PRINT\@: Stampa tutti i DATA degli elementi della lista in ordine di apparizione.
    \item SORT\@: Ordina in modo crescente gli elementi della lista in base a DATA\@.
    \item REV\@: Inverte gli elementi della lista.
\end{itemize}

I comandi da eseguire sono contenuti in una variabile \textit{listInput} di tipo string. 
Sono separati da ''\texttildelow'' e possono essere presenti spazi prima e dopo la tilde.