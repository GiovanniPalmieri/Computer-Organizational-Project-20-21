\section{Descrizione del ciclo principale}

Il compito del ciclo principale \textit{mainLoop} è quello di leggere 
l'array in input \textit{listInput} e in base ad esso chiamare le funzioni relative ai comandi.

\subsection{Riconoscimento input}
Dato che non esistono comandi diversi che abbiano la stessa lettera iniziale,
distinguiamo i comandi in base alla prima lettera.
\\
\begin{table}[H]
\begin{center}
\begin{tabular}{|c|c|}
    \hline
    Carattere & Significato \\
    \hline
    A & ADD\{\ldots\} \\
    \hline
    D & DEL\{\ldots\} \\
    \hline
    P & PRINT \\
    \hline
    S & SORT \\
    \hline
    R & REV \\
    \hline
\end{tabular}
\caption{Riconoscimento comandi}
\label{tab:commandIdentification}
\end{center}
\end{table}

Dobbiamo però distinguere le lettere relative ai nomi dei comandi dai valori dei parametri 
dei comandi, esempio:\\
\centerline{\textcircled{A}DD\{\textcircled{A}\}} \\
%
Per evitare questo tipo di letture scorrette evitiamo direttamente di esaminare i parametri
dei comandi. Per farlo ci avvaliamo del fatto che:
\begin{enumerate}
    \item Il primo carattere che troviamo della stringa di input, non può essere il 
    parametro di un comando.
    \item Dato che non sono ammessi spazi all'interno dei comandi sappiamo la lunghezza di 
    ogni comando.
\end{enumerate}
%
Perciò ogni volta che riconosciamo la prima lettera di un comando, spostiamo
il puntatore alla lista di input alla fine del comando. \\
\begin{table}[H]
\begin{center}
\begin{tabular}{|c|c|}
    \hline
    Comando & Valore da sommare \\
    \hline 
    ADD & 6 + 1 \\
    \hline
    DEL & 6 + 1 \\
    \hline
    PRINT & 5 + 1 \\
    \hline
    SORT & 4 + 1 \\
    \hline
    REV & 3 + 1 \\
    \hline
\end{tabular}
\caption{Valori da sommare per evitare 
letture incorrette}
\label{tab:valoriDaSommare}
\end{center}
\end{table}
%
Il "+1" in Tabella \ref{tab:valoriDaSommare} ci 
permette di risparmiare la validazione di un char,
dato che i comandi sono divisi da almeno un ''\texttildelow''.

\subsection{Chiamata delle operazioni}
Ogni volta che viene riconosciuto un comando, viene eseguita una jump al blocco di codice che gestisce la
chiamata al relativo comando.\\
(\textit{callAdd}, \textit{callDel}, \textit{callSort}, \textit{callRev}, \textit{callPrint})

Prima di ogni chiamata vengono salvati i registri nello stack, e in caso di comandi
con un parametro (\textit{ADD}, \textit{DEL}), il parametro viene messo in \textbf{a0}.
Poi viene incrementato il puntatore alla stringa di input secondo la Tabella \ref{tab:valoriDaSommare}
