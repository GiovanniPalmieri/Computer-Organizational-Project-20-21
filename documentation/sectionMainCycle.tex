\section{Descrizione del ciclo principale}

Il compito del ciclo principale \textit{mainLoop} è quello di leggere 
l'array di input \textit{listInput} e in base ad esso chiamare le funzioni relative ai comandi

\subsection{Riconoscimento input}
Un comando viene riconosciuto semplicemente dalla prima lettera, dato che non esitono comandi che abbiano la stessa lettera iniziale.
\\
\begin{table}[H]
\begin{center}
\begin{tabular}{|c|c|}
    \hline
    Carattere & Significato \\
    \hline
    A & ADD\{...\} \\
    \hline
    D & DEL\{...\} \\
    \hline
    P & PRINT \\
    \hline
    S & SORT \\
    \hline
    R & REV \\
    \hline
\end{tabular}
\caption{Riconoscimento comandi}
\label{tab:commandIdentification}
\end{center}
\end{table}

Bisogna però considerare il caso in cui, 
una lettera relativa ad un comando sia in realtà il parametro di un comando, esempio:\\
\centerline{\textcircled{A}DD\{\textcircled{A}\}} \\


Per distunguere questi due casi ci avvaliamo del fatto che:
\begin{enumerate}
    \item Il primo carattere che troviamo,relativo ad un comando, non può essere il parametro di un comando.
    \item Per ogni comando sappiamo quanto è lungo, dato che non sono ammessi spazi all'interno dei comandi.
\end{enumerate}

Perciò ogni volta che leggiamo un comando,
 spostiamo il puntatore alla lista di input alla fine del comando. \\
\begin{table}[H]
\begin{center}
\begin{tabular}{|c|c|}
    \hline
    Comando & Valore da sommare \\
    \hline 
    ADD & 6 + 1 \\
    \hline
    DEL & 6 + 1 \\
    \hline
    PRINT & 5 + 1 \\
    \hline
    SORT & 4 + 1 \\
    \hline
    REV & 3 + 1 \\
    \hline
\end{tabular}
\caption{Tabella valori da sommare per evitare 
letture incorrette}
\label{tab:valoriDaSommare}
\end{center}
\end{table}

Il "+1" in Tabella \ref{tab:valoriDaSommare} ci 
permette di risparmiare la validazione di un char,
 dato che i comandi sono divisi da almeno un "\~" 
 e eventuali spazi.

\subsection{Chiamata delle operazioni}
Ogni volta che viene riconosciuto un comando, viene eseguita una jump al blocco di codice che gestisce la
chiamata al relativo comando.\\
(\textit{callAdd}, \textit{callDel}, \textit{callSort}, \textit{callRev}, \textit{callPrint})

Prima di ogni chiamata vengono salvati i registri nello stack, e in caso di comandi
con un parametro (\textit{ADD}, \textit{DEL}), il parametro viene messo in \textbf{a0}.
Poi viene incrementato il puntatore alla stringa di input secondo la Tabella \ref{tab:valoriDaSommare}
