\section{Descrizione Delete}

\paragraph{Descrizione}
Il comando Del, rimuove il primo nodo il cui valore è uguale al parametro passato,
se esiste, altrimenti non altera la lista.

\paragraph{Implementazione}
La funzione scorre tutta la lista finchè non arriva alla fine, oppure finchè non trova
un nodo con il valore da cancellare. Una volta trovato il nodo si distunguono due casi:
\begin{itemize}
    \item Il nodo è l'ultimo nodo della lista, in questo caso si imposta il valore 
    del penultimo nodo a -1, in questo modo diventa l'ultimo.
    \item Il nodo non è l'ultimo nodo della lista, in questo caso dobbiamo impostare 
    i puntatori del nodo precedente e del nodo successivo come mostrato in Figura~
\end{itemize}