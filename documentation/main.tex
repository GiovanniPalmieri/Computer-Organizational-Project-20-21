\documentclass[a4paper,12pt]{article}
\usepackage[utf8]{inputenc}
\usepackage[italian]{babel}

\title{Relazione progetto Architettura degli elaboratori}
\author{Giovanni Palmieri (7006086)}

\begin{document}

\maketitle

\tableofcontents


\section{Introduzione}
Introduzione


\section{Descrizione del ciclo principale}

Il compito del ciclo principale \textit{mainLoop} è quello di leggere 
l'array di input \textit{inputText} e in base ad esso chiamare le funzioni relative ai comandi

\subsection{Riconoscimento input}
Un comando viene riconosciuto semplicemente dalla prima lettera, dato che non esitono comandi che abbiano la stessa lettera iniziale.
Perciò un carattere \textbf{A} nella stringa di input può voler dire due cose:
\begin{enumerate}
    \item La prima lettera del comando ADD\{...\}
    \item Il dato di un comando ...\{A\}
\end{enumerate}

Per distinguere i due casi il programma quando interpreta la prima lettera di un comando {A,D,L,S,R} 

\end{document}