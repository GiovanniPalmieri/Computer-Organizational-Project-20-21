\section{Add}
\subsection{Implementazione}
Dato che la lista non ha un puntatore alla coda, ma solo un puntatore alla testa è necessario
trovare l'ultimo elemento. Si può individuare perchè è l'unico elemento con PAHED uguale a -1.
Una volta individuato l'ultimo elemento viene creato il nuovo nodo con PBACK uguale all'
indirizzo dell'ultimo elemento e viene aggiornato il PAHEAD dell'ultimo elemento con 
l'indirizzo del nuovo elemento.
L'allocazione dello spazio in memoria viene effettuata dalla funzione 
\textit{getPseudoRandom} descritta nella sezione successiva.
Nel caso in cui la lista non contenga elementi e quindi il punatore al primo elemento sia
uguale a 0, basta creare il nuovo elemento con PBACK e PAHEAD uguali a -1 e impostare
il puntatore al primo elemento all'indirizzo del nuovo elemento.

\subsection{Funzione getPseudoRandom}
Per prima cosa dobbiamo generare un numero casuale nel range della memoria del programma,
per farlo usiamo l'LFSR (Linear-Feedback Shift Register). 
L'LFSR permette di generare un numero pseudo casuale eseguendo uno shift a destra e sostituendo il bit vuoto 
con il risultato di una XOR dei bit.
Per funzionare la funzione ha bisogno di un \textbf{seed}, ossia di un valore iniziale, che viene 
impostato la prima volta che si chiama la funzione.
 Poi il \textbf{newBit} vine calcolato eseguendo lo XOR tra il primo,
terzo, quarto e quinto bit. A questo punto si esegue lo shift a destra del valore corrente e si sostituisce il bit 
più valente con il \textbf{newBit}. Quest'operazione viene fatta su 16 bit, dato che risv-v usa 4 byte per indirizzare la memoria,
 ci restano i due byte più valenti da riempire. Per evitare overflow e/o sovrascritture di dati esistenti
impostiamo i due byte a 0x00001.

\subsection{Miglioramenti e variazioni}
L'aggiunta del nuovo elento potrebbe essere resa più efficente salvando il puntatore 
all'ultimo elemento, in questo modo non ci sarebbe bisogno di scorrere tutta la lista
per trovarlo. Lo svantaggio di questa implementazione è che si deve aggiornare il 
puntatore all'ultimo elemento ogni volta che l'ultimo elemento cambia. 