\section{Descrizione Add}
\paragraph{Descrizione}
Il comando Add, aggiunge un nodo in coda alla lista, (ossia come ultimo elemento). 
Dato che la lista è una lista concatenata deve essere allocato dinamicamente lo spazione per l'ultimo elemento.

\paragraph{Implementazione}
In base all'implementazione della lista concatenata l'operazione per eseguire la Add è diversa.
Nel caso del programma non abbiamo un puntatore alla coda, perciò la prima fase della Add è trovare l'ultimo elemento.
La ricerca dell'ultimo elemento viene effettuata scorrendo la lista, finchè non si trova un nodo che non ha il punatore al prossimo elemento.
A questo punto deve essere allocata un area libera di memoria, e ci viene messo il dato da aggiungere.
Dopo di che viene aggiornato il puntatore all'elemento successivo del penutlimo elemento, 
e i puntatori all'elemento precedente e successivo dell'ultimo elemento.
È inoltre necessario distunguere il caso in cui la lista sia vuota e inizializzare il puntatore al primo elmento.

\subsection{Funzione getPseudoRandom}
\paragraph{Descrizione}
Come abbiamo detto nella descrizione dell'operazione è necessario trovare un area libera di memoria.
Per fare ciò usiamo un generatore di numeri pseudorandomici. 