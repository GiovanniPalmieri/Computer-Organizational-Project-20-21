\section{Descrizione Add}
\paragraph{Descrizione}
Il comando Add, aggiunge un nodo in coda alla lista, (ossia come ultimo elemento). 
Dato che la lista è una lista concatenata deve essere allocato dinamicamente lo spazione per l'ultimo elemento.

\paragraph{Implementazione}
In base all'implementazione della lista concatenata l'operazione per eseguire la Add è diversa.
Nel caso del programma non abbiamo un puntatore alla coda, perciò la prima fase della Add è trovare l'ultimo elemento.
La ricerca dell'ultimo elemento viene effettuata scorrendo la lista, finchè non si trova un nodo che non ha il punatore al prossimo elemento.
A questo punto deve essere allocata un area libera di memoria, e ci viene messo il dato da aggiungere.
Dopo di che viene aggiornato il puntatore all'elemento successivo del penutlimo elemento, 
e i puntatori all'elemento precedente e successivo dell'ultimo elemento.
È inoltre necessario distunguere il caso in cui la lista sia vuota e inizializzare il puntatore al primo elmento.

\subsection{Funzione getPseudoRandom}
\paragraph{Descrizione}
Come abbiamo detto nella descrizione dell'operazione è necessario trovare un area libera di memoria.
Per fare ciò usiamo un generatore di numeri pseudorandomici. 
\paragraph{Implementazione}
Per generare il numero pseduo random usiamo l'LFSR (Linear-Feedback Shift Register). 
Che permette di generare un numero pseudo casuale eseguendo uno shift a destra e sostituendo il bit vuoto 
con il risultato di una XOR dei bit.
Per funzionare abbiamo bisogno di un \textbf{seed}, ossia di un valore iniziale. La prima volta che viene chiamata il seed 
viene impostato come valore corrente. Poi il \textbf{newBit} vine calcolato eseguendo lo xor tra il primo,
terzo, quarto e quinto bit. A questo punto si esegue lo shift a destra del valore corrente e si sostituisce il bit 
più valente con il \textbf{newBit}. Quest'operazione viene fatta su 16 bit, dato che risv-v usa 4 byte per indirizzare la memoria,
 ci restano i due byte più valenti da riempire. Per evitare overflow e/o sovrascritture di dati esistenti
impostiamo i due byte a 0x00001.   
\\
\begin{lstlisting}[language=java,caption={Codice java algoritmo generazione psudo-casuale}, captionpos=b]

    bool firstInvocation = true;
    String seed = "0001011000111010";
    String lfsr;
    private void getPseudoRandom(){
        if(firstInvocation){
            lfsr = seed;
        }
        char newBit = 
    }
    
\end{lstlisting}